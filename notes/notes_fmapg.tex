%% if some package prevents compilation, remove it

\documentclass[a4paper, 10pt]{article}
\usepackage{fullpage}
\usepackage{titling}
\usepackage{graphicx}
\usepackage{xcolor}
\usepackage{amsmath}
\usepackage{amssymb}
\usepackage{amsthm}
\usepackage{hyperref}
\usepackage{IEEEtrantools} 
\usepackage{bbm}
\usepackage{lineno}
\usepackage{multirow}
\usepackage{longtable}
\usepackage{makecell}
\usepackage{lipsum}
\usepackage{etoolbox} %% <- for \pretocmd and \apptocmd
\usepackage{algorithm}
\usepackage[noend]{algpseudocode}

\setlength{\droptitle}{-6em}   % This is your set screw, for shifting up the title
\renewcommand{\baselinestretch}{1.1}
\setlength{\parskip}{0.5em}

\author{Shivam and Sharan
  \\ September 2021}
\date{}
\title{FMA-PG Notes}

\makeatletter %% <- make @ usable in macro names
\newcommand*\linenomathpatch{\@ifstar{\linenomathpatch@AMS}{\linenomathpatch@}}
\newcommand*\linenomathpatch@[1]{
  \expandafter\pretocmd\csname #1\endcsname {\linenomathWithnumbers}{}{}
  \expandafter\pretocmd\csname #1*\endcsname{\linenomathWithnumbers}{}{}
  \expandafter\apptocmd\csname end#1\endcsname {\endlinenomath}{}{}
  \expandafter\apptocmd\csname end#1*\endcsname{\endlinenomath}{}{}
}
\newcommand*\linenomathpatch@AMS[1]{
  \expandafter\pretocmd\csname #1\endcsname {\linenomathWithnumbersAMS}{}{}
  \expandafter\pretocmd\csname #1*\endcsname{\linenomathWithnumbersAMS}{}{}
  \expandafter\apptocmd\csname end#1\endcsname {\endlinenomath}{}{}
  \expandafter\apptocmd\csname end#1*\endcsname{\endlinenomath}{}{}
}
\let\linenomathWithnumbersAMS\linenomathWithnumbers
\patchcmd\linenomathWithnumbersAMS{\advance\postdisplaypenalty\linenopenalty}{}{}{}
\makeatother %% revert @
\linenomathpatch{IEEEeqnarray}
\linenomathpatch{equation}
\linenomathpatch*{gather}
\linenomathpatch*{multline}
\linenomathpatch*{align}
\linenomathpatch*{alignat}
\linenomathpatch*{flalign}

\DeclareMathOperator{\E}{\mathbb{E}}

\begin{document}
\maketitle
\vspace{-2cm}
% \linenumbers

\section{Softmax PPO with Tabular Parameterization}

\subsection{Closed Form Update with Direct Representation}
We will consider direct functional representation with tabular parameterization, i.e. $\pi \equiv p^\pi$ is essentially an $\mathrm{S} \times \mathrm{A}$ table satisfying the constraints
\begin{IEEEeqnarray*}{rCll}
  \sum_a p^\pi(a | s) &=& 1, & \quad \forall s \in \mathcal{S} \\
  p^\pi(a | s) &\geq& 0, & \quad \forall s \in \mathcal{S}, \; \forall a \in \mathcal{A}.
\end{IEEEeqnarray*}
Our goal is to find the closed form solution to the following optimization problem (from Eq. 6, Sharan et al., 2021):
\begin{equation}
  \pi_{t+1} = \arg\max_{\pi \in \Pi} \left[ \sum_s d^{\pi_t}(s) \sum_a p^{\pi_t}(a | s) \left(A^{\pi_t}(s, a) + \frac{1}{\eta} \right) \log \frac{p^\pi(s, a)}{p^{\pi_t}(s, a)} \right], \label{eq: optim_problem_sppo}
\end{equation}
subject to the constraints on $p^\pi$ given above. 

We begin by formulating this problem using Lagrange multipliers $\lambda_s$, $\lambda_{s, a}$ for all states $s$ and actions $a$:
\begin{IEEEeqnarray}{rCl}
  \mathcal{L}(p^\pi, \lambda_s, \lambda_{s, a}) &=& \sum_s d^{\pi_t}(s) \sum_a p^{\pi_t}(a | s) \left(A^{\pi_t}(s, a) + \frac{1}{\eta} \right) \log \frac{p^\pi(a | s)}{p^{\pi_t}(a | s)} \nonumber \\
  && - \sum_{s, a} \lambda_{s, a} p^\pi(a | s) - \sum_s \lambda_{s} \bigg( \sum_a p^\pi(a | s) - 1 \bigg),
\end{IEEEeqnarray}
where we abused the notation by using $\lambda_s$ to represent the set $\{\lambda_s\}_{s \in \mathcal{S}}$ and $\lambda_{s, a}$ to represent the set $\{\lambda_{s, a}\}_{s, a \in \mathcal{S} \times \mathcal{A}}$. The KKT conditions (Theorem 12.1, Nocedal and Wright, 2006) for this constrained optimization problem can be written as:
\begin{align}
  \nabla_{p^\pi(x, b)} \mathcal{L}(p^\pi, \lambda_s, \lambda_{s, a}) &= 0, \quad \forall x \in \mathcal{S}, \; \forall b \in \mathcal{A} \tag{C1} \label{eq: KKT1} \\
  \sum_a p^\pi(a | s) &= 1, \quad \forall s \in \mathcal{S} \tag{C2} \label{eq: KKT2} \\
  p^\pi(a | s) &\geq 0, \quad \forall s \in \mathcal{S}, \; \forall a \in \mathcal{A} \tag{C3} \label{eq: KKT3} \\
  \lambda_s &\geq 0, \quad \forall s \in \mathcal{S} \tag{C4} \label{eq: KKT4} \\
  \lambda_{s} \bigg( \sum_a p^\pi(a | s) - 1 \bigg) &= 0, \quad \forall s \in \mathcal{S} \tag{C5} \label{eq: KKT5} \\
  \lambda_{s, a} p^\pi(a | s) &= 0, \quad \forall s \in \mathcal{S}, \; \forall a \in \mathcal{A}. \tag{C6} \label{eq: KKT6}
\end{align}

Let us now try to solve this system. Solving the first equation for an arbitrary state-action pair $(x, b)$, gives us:
\begin{IEEEeqnarray}{lrCl}
  & \nabla_{p^\pi(b | x)} \mathcal{L}(p^\pi, \lambda_s, \lambda_{s, a}) &=& d^{\pi_t}(x) p^{\pi_t}(b|x) \left( A^{\pi_t}(x, b) + \frac{1}{\eta} \right) \frac{1}{p^\pi(b|x)} - \lambda_{x, b} - \lambda_x = 0 \nonumber \\
  \Rightarrow & p^\pi(b | x) &=& \frac{d^{\pi_t}(x) p^{\pi_t}(b|x) (1 + \eta A^{\pi_t}(x, b))}{\eta (\lambda_x + \lambda_{x, b})}. \label{eq: lagrangian_derivative_sppo}
\end{IEEEeqnarray}
Let us set 
\begin{equation}
  \lambda_{s, a} = 0, \quad \forall s \in \mathcal{S}, \; \forall a \in \mathcal{A}.
\end{equation}
Combining Eq. \ref{eq: lagrangian_derivative_sppo} with the second KKT condition gives us
\begin{equation}
  \lambda_s = \frac{1}{\eta} \sum_a d^{\pi_t}(s) p^{\pi_t}(a|s) (1 + \eta A^{\pi_t}(s, a)).
\end{equation}
Therefore, with the additional assumption $d^{\pi_t}(s) > 0$, $p^\pi(a | s)$ becomes
\begin{equation}
  p^\pi(a | s) = \frac{p^{\pi_t}(a|s) (1 + \eta A^{\pi_t}(s, a))}{\sum_b p^{\pi_t}(b|s) (1 + \eta A^{\pi_t}(s, b))}.
\end{equation}
Note that $d^{\pi_t}(s), p^{\pi_t}(a|s) \geq 0$ for any state-action pair, since they are proper measures. All that remains is to ensure that
\begin{equation*}
  1 + \eta A^{\pi_t}(s, a) \geq 0
\end{equation*}
to satisfy the third and fourth KKT conditions. But how to do that? One straightforward way is to define $p^\pi(a | s) = 0$ whenever $1 + \eta A^{\pi_t}(s, a) < 0$, and accordingly re-define $\lambda_s$. This gives us the final solution to our original optimization problem (Eq. \ref{eq: optim_problem_sppo}):
\begin{equation}
  \pi_{t+1} = p^\pi(s, a) = \frac{p^{\pi_t}(a|s) \max(1 + \eta A^{\pi_t}(s, a), 0)}{\sum_b p^{\pi_t}(b|s) \max(1 + \eta A^{\pi_t}(s, b), 0)}.
\end{equation}
\textbf{However, it leaves us one last problem to deal with:} Is it always true that given any state $s$, there always exists atleast one action $a$, such that $1 + \eta A^{\pi_t}(s, a) \geq 0$? Because otherwise, we would fail to satisfy the second KKT condition. Maybe, we can put a condition on $\eta$ in order to fulfill this constraint.

\subsection{Gradient of the Loss Function with Softmax Policy Representation}
Consider the softmax policy representation
\begin{equation}
  p^\pi(b | x) = \frac{e^{\theta(x, b)}}{\sum_c e^{\theta(x, c)}}, \label{eq: softmax}
\end{equation}
where $\theta(x, b)$s for all state-action pairs $(x, b)$ are action preferences maintained in a table (tabular parameterization). We will use gradient ascent to approximately solve Eq. \ref{eq: optim_problem_sppo}; to do that, the quantity of interest is
\begin{align}
  \nabla_{\theta(s, a)} \ell^{\pi_t} &= \sum_{x \in \mathcal{S}} \sum_{b \in \mathcal{A}} \left[ \nabla_{\theta(s, a)} p^\pi(b | x) \right] \left[ \nabla_{p^\pi(b | x)} \ell^{\pi_t} \right] \tag*{(using total derivative)} \\
  &= \sum_{x, b} \Big[ \mathbb{I}(x = s) \Big( \mathbb{I}(b = a) - p^\pi(a | x) \Big) p^\pi(b | x) \Big] \left[ d^{\pi_t}(x) p^{\pi_t}(b|x) \left( A^{\pi_t}(x, b) + \frac{1}{\eta} \right) \frac{1}{p^\pi(b|x)} \right] \nonumber \\
  &= \E_{X \sim d^{\pi_t}, B \sim p^{\pi_t}(\cdot | X)} \left[ \mathbb{I}(X = s) \Big( \mathbb{I}(B = a) - p^\pi(a | x) \Big) \left( A^{\pi_t}(X, B) + \frac{1}{\eta} \right) \right] \\
  &= d^{\pi_t}(s) \sum_b \Big( \mathbb{I}(b = a) - p^\pi(a | s) \Big) p^{\pi_t}(b|s) \left( A^{\pi_t}(s, b) + \frac{1}{\eta} \right) \nonumber \\
  &= d^{\pi_t}(s) \left[ p^{\pi_t}(a|s) \left( A^{\pi_t}(s, a) + \frac{1}{\eta} \right) - p^\pi(a | s) \sum_b p^{\pi_t}(b|s) \left(A^{\pi_t}(s, b) + \frac{1}{\eta} \right) \right] \nonumber \\
  &= d^{\pi_t}(s) \left[ p^{\pi_t}(a|s) \left( A^{\pi_t}(s, a) + \frac{1}{\eta} \right) - \frac{p^\pi(a | s)}{\eta} \right]. \nonumber
\end{align}
Then, we can simply update the inner loop of FMA-PG (Algorithm 1, Sharan et al., 2021) via gradient ascent:
\begin{equation}
  \theta_{s, a} = \theta_{s, a} + \alpha d^{\pi_t}(s) \left[ p^{\pi_t}(a|s) \left( A^{\pi_t}(s, a) + \frac{1}{\eta} \right) - \frac{p^\pi(a | s)}{\eta} \right].
\end{equation}

\section{MDPO}
\subsection{Closed Form Update with Direct Parameterization}
The paper (Sharan et al., 2021) considers the direct representation along with tabular parameterization of the policy, albeit with a small change in notation as compared to the previous section: $\pi(a|s) \equiv p^\pi(a|s, \theta)$. However, since this notation is more cumbersome, we will stick with our old notation: $\pi(a|s) \equiv p^\pi(a|s)$. The constraints on the parameters $p^\pi(s, a)$ are the same as before: $\sum_a p^\pi(a | s) = 1, \; \forall s \in \mathcal{S}$; and $p^\pi(a | s) \geq 0, \; \forall s \in \mathcal{S}, \; \forall a \in \mathcal{A}$. Our goal, this time, is to solve the following optimization problem (from Eq. 9, Sharan et al., 2021)
\begin{equation}
  \pi_{t+1} = \arg\max_{\pi \in \Pi} \left[ \sum_s d^{\pi_t}(s) \sum_a p^{\pi_t}(a|s) \left( Q^{\pi_t}(s, a) \frac{p^\pi(a | s)}{p^{\pi_t}(a | s)} - \frac{1}{\eta} D_\phi (p^\pi(\cdot | s), p^{\pi_t}(\cdot | s)) \right) \right], \label{eq: optim_problem_mdpo}
\end{equation}
with the mirror map as the negative entropy (Eq. 5.27, Beck and Teboulle, 2002). This particular choice of the mirror map simplifies the Bregman divergence as follows
\begin{equation}
  D_\phi (p^\pi(\cdot | s), p^{\pi_t}(\cdot | s)) = \text{KL}(p^\pi(\cdot | s) \| p^{\pi_t}(\cdot | s)) := \sum_a p^\pi(a | s) \log \frac{p^\pi(a | s)}{p^{\pi_t}(a | s)}.
\end{equation}
The optimization problem (Eq. \ref{eq: optim_problem_mdpo}) then simplifies to
\begin{equation}
  \pi_{t+1} = \arg\max_{\pi \in \Pi} \left[ \sum_s d^{\pi_t}(s) \sum_a p^{\pi_t}(a|s) \left( Q^{\pi_t}(s, a) \frac{p^\pi(a | s)}{p^{\pi_t}(a | s)} - \frac{1}{\eta} \sum_{a'} p^\pi(a' | s) \log \frac{p^\pi(a' | s)}{p^{\pi_t}(a' | s)} \right) \right].
\end{equation}

Proceeding analogously to the previous section, we use Lagrange multipliers $\lambda_s$, $\lambda_{s, a}$ for all states $s$ and actions $a$ to obtain the function
\begin{IEEEeqnarray}{rCl}
  \mathcal{L}(p^\pi, \lambda_s, \lambda_{s, a}) &=& \sum_s d^{\pi_t}(s) \sum_a p^{\pi_t}(a|s) Q^{\pi_t}(s, a) \frac{p^\pi(a | s)}{p^{\pi_t}(a | s)} - \frac{1}{\eta} \sum_s d^{\pi_t}(s) \sum_{a'} p^\pi(a' | s) \log \frac{p^\pi(a' | s)}{p^{\pi_t}(a' | s)} \nonumber \\
  && - \sum_{s, a} \lambda_{s, a} p^\pi(a | s) - \sum_s \lambda_{s} \bigg( \sum_a p^\pi(a | s) - 1 \bigg).
\end{IEEEeqnarray}
The KKT conditions are exactly the same as before (Eq. \ref{eq: KKT1} to Eq. \ref{eq: KKT6}).

Again, we begin by solving the first KKT condition:
\begin{IEEEeqnarray}{lrCl}
  & \nabla_{p^\pi(b | x)} \mathcal{L}(p^\pi, \lambda_s, \lambda_{s, a}) &=& d^{\pi_t}(x) p^{\pi_t}(b|x) \frac{Q^{\pi_t}(x, b)}{p^{\pi_t}(b | x)} - \frac{d^{\pi_t}(x)}{\eta} \left[ \log \frac{p^\pi(b | x)}{p^{\pi_t}(b | x)} + 1 \right] - \lambda_{x, b} - \lambda_x \nonumber \\
  &&=& \frac{d^{\pi_t}(x)}{\eta} \left[ \eta Q^{\pi_t}(x, b) - \log \frac{p^\pi(b | x)}{p^{\pi_t}(b | x)} - 1 - \frac{\eta (\lambda_{x, b} + \lambda_x)}{d^{\pi_t}(x)} \right] \nonumber \\
  &&=& 0 \nonumber \\
  \Rightarrow & \log \frac{p^\pi(b | x)}{p^{\pi_t}(b | x)} &=& \eta Q^{\pi_t}(x, b) - \frac{\eta (\lambda_{x, b} + \lambda_x)}{d^{\pi_t}(x)} - 1 \nonumber \\
  \Rightarrow & p^\pi(b | x) &=& p^{\pi_t}(b | x) \cdot e^{\eta Q^{\pi_t}(x, b)} \cdot e^{- \frac{\eta (\lambda_{x, b} + \lambda_x)}{d^{\pi_t}(x)} - 1}, \label{eq: lagrangian_derivative_mdpo}
\end{IEEEeqnarray}
where in the fourth line, we made the assumption that $d^{\pi_t}(x) > 0$ for all states $x$. We again set
\begin{equation}
  \lambda_{s, a} = 0, \quad \forall s \in \mathcal{S}, \; \forall a \in \mathcal{A}.
\end{equation}
And, we put Eq. \ref{eq: lagrangian_derivative_mdpo} in the second KKT condition to get
\begin{equation}
  e^{- \frac{\eta \lambda_x}{d^{\pi_t}(x)} - 1} = \left( \sum_b p^{\pi_t}(b | x) \cdot e^{\eta Q^{\pi_t}(x, b)} \right)^{-1}.
\end{equation}
Therefore, we obtain
\begin{equation}
  p^\pi(a | s) = \frac{p^{\pi_t}(a | s) \cdot e^{\eta Q^{\pi_t}(s, a)}}{\sum_b p^{\pi_t}(b | s) \cdot e^{\eta Q^{\pi_t}(s, b)}}.
\end{equation}

\textbf{This leaves one last problem:} Can we ensure that $\lambda_s \geq 0$ for all states $s$? If not, then the fourth KKT condition cannot be satisfied. Maybe, we can set the stepsize $\eta$ in such a way, such that this constraint is always fulfilled.

\subsection{Gradient of the Loss Function with Softmax Policy Representation}
We again take the softmax policy representation given by Eq. \ref{eq: softmax}, and compute $\nabla_{\theta(s, a)} \ell^{\pi_t}$ for the MDPO loss (we substitute $Q^{\pi_t}$ with $A^{\pi_t}$ in this calculation):
\begin{align}
  \nabla_{\theta(s, a)} \ell^{\pi_t} &= \sum_{x, b} \left[ \nabla_{\theta(s, a)} p^\pi(b | x) \right] \left[ \nabla_{p^\pi(b | x)} \ell^{\pi_t} \right] \tag*{(using total derivative)} \\
  &= \sum_{x, b} \Big[ \mathbb{I}(x = s) \Big( \mathbb{I}(b = a) - p^\pi(a | x) \Big) p^\pi(b | x) \Big] \left[ \frac{d^{\pi_t}(x)}{\eta} \left( \eta A^{\pi_t}(x, b) - \log \frac{p^\pi(b | x)}{p^{\pi_t}(b | x)} - 1 \right) \right] \nonumber \\
  &= \frac{d^{\pi_t}(s)}{\eta} \sum_b \Big( \mathbb{I}(b = a) - p^\pi(a | s) \Big) p^\pi(b | s) \left[ \eta A^{\pi_t}(s, b) - \log \frac{p^\pi(b | s)}{p^{\pi_t}(b | s)} - 1 \right] \nonumber \\
  &= \frac{d^{\pi_t}(s)}{\eta} p^\pi(a | s) \left[ \eta A^{\pi_t}(s, a) - \eta \sum_b p^\pi(b|s) A^{\pi_t}(s, b) - \log \frac{p^\pi(a | s)}{p^{\pi_t}(a | s)} + \text{KL}(p^\pi(\cdot | s) \| p^{\pi_t}(\cdot | s)) \right], \nonumber
\end{align}
where in the last line, we used the fact that
\begin{equation*}
  \sum_b p^\pi(b | s) \left[ \eta A^{\pi_t}(s, b) - \log \frac{p^\pi(b | s)}{p^{\pi_t}(b | s)} - 1 \right] = \eta \sum_b p^\pi(b|s) A^{\pi_t}(s, b) - \text{KL}(p^\pi(\cdot | s) \| p^{\pi_t}(\cdot | s)) - 1.
\end{equation*}

\section{TRPO}
At each step of the policy update, TRPO (Eq. 14, Schulman et al., 2015) solves the following problem:
\begin{equation}
  \max_\theta \; \underbrace{\sum_s d^{\pi_t}(s) \sum_a p^{\pi_\theta}(a | s) Q^{\pi_t}(s, a)}_{=: \mathcal{J}} \qquad \text{subject to } \underbrace{\sum_s d^{\pi_t}(s) \cdot \text{KL}(p^{\pi_t}(\cdot | s) \| p^{\pi_\theta}(\cdot | s))}_{=: \mathcal{C}} \leq \delta.  
\end{equation}
Unlike the sPPO and the MDPO updates, most likely (not absolutely sure though) an analytical solution cannot be derived for this update (since it would require solving a system of non-trivial non-linear equations; to see this, try writing the KKT conditions for this constrained optimization problem). Therefore, we will use gradient based methods to approximately solve this problem. From Appendix C of Schulman et al. (2015), the descent direction is given by $s \approx A^{-1} g$ where the vector $g$ is defined as $g_{(s, a)} := \frac{\partial}{\partial \theta(s, a)} \mathcal{J}$, and the matrix $A$ is defined as $A_{(s, a), (s', a')} := \frac{\partial}{\partial \theta(s, a)} \frac{\partial}{\partial \theta(s', a')} \mathcal{C}$. We compute this direction assuming a softmax policy (Eq. \ref{eq: softmax}). The vector $g$ can be readily calculated as
\begin{align}
  \frac{\partial}{\partial \theta(s, a)} \mathcal{J} &= \sum_x d^{\pi_t}(x) \sum_b Q^{\pi_t}(x, b) \frac{\partial p^{\pi_\theta}(b | x)}{\partial \theta(s, a)} \nonumber \\
  &= \sum_x d^{\pi_t}(x) \sum_b Q^{\pi_t}(x, b) \mathbb{I}(x = s) \Big( \mathbb{I}(b = a) - p^{\pi_\theta}(a | x) \Big) p^{\pi_\theta}(b | x) \nonumber \\
  &= \sum_x d^{\pi_t}(x) \mathbb{I}(x = s) \left[ \sum_b \mathbb{I}(b = a) p^{\pi_\theta}(b | x) Q^{\pi_t}(x, b) - p^{\pi_\theta}(a | x) \sum_b p^{\pi_\theta}(b | x) Q^{\pi_t}(x, b) \right] \nonumber \\
  &= d^{\pi_t}(s) p^{\pi_\theta}(a | s) \left[ Q^{\pi_t}(s, a) - \sum_b p^{\pi_\theta}(b | s) Q^{\pi_t}(s, b) \right]. \label{eq: trpo_gradient}
\end{align}
For calculating the matrix $A$, note that
\begin{equation*}
  \frac{\partial \mathcal{C}}{\partial p^{\pi_\theta}(b | x)} = \frac{\partial}{\partial p^{\pi_\theta}(b | x)} \sum_s d^{\pi_t}(s) \sum_a p^{\pi_t}(a | s) \log \frac{p^{\pi_t}(a | s)}{p^{\pi_\theta}(a | s)} = - d^{\pi_t}(x) \frac{p^{\pi_t}(b | x)}{p^{\pi_\theta}(b | x)}.
\end{equation*}
Then, using the law of total derivative, gives us
\begin{align}
  \frac{\partial}{\partial \theta(s, a)} \mathcal{C} &= \sum_{x, b} \frac{\partial p^{\pi_\theta}(b | x)}{\partial \theta(s, a)} \cdot \frac{\partial \mathcal{C}}{\partial p^{\pi_\theta}(b | x)} \nonumber \\
  &= - \sum_{x, b} \mathbb{I}(x = s) \Big( \mathbb{I}(b = a) - p^{\pi_\theta}(a | x) \Big) p^{\pi_\theta}(b | x) \cdot d^{\pi_t}(x) \frac{p^{\pi_t}(b | x)}{p^{\pi_\theta}(b | x)} \nonumber \\
  &= - d^{\pi_t}(s) \sum_b \Big( \mathbb{I}(b = a) - p^{\pi_\theta}(a | s) \Big) p^{\pi_t}(b | s) \nonumber \\
  &= - d^{\pi_t}(s) \left[ \sum_b \mathbb{I}(b = a) p^{\pi_t}(b | s) - p^{\pi_\theta}(a | s) \sum_b p^{\pi_t}(b | s) \right] \nonumber \\
  &= - d^{\pi_t}(s) \Big[ p^{\pi_\theta}(a | s) - p^{\pi_t}(a | s) \Big].
\end{align}
Finally, using the above result yields
\begin{IEEEeqnarray}{lrCl}
  & \frac{\partial}{\partial \theta(s, a)} \frac{\partial}{\partial \theta(s', a')} \mathcal{C} &=& \frac{\partial}{\partial \theta(s, a)} d^{\pi_t}(s') \Big[ p^{\pi_\theta}(a | s) - p^{\pi_t}(a | s) \Big] \nonumber \\
  &&=& d^{\pi_t}(s') \cdot \frac{\partial}{\partial \theta(s, a)} p^{\pi_\theta}(a' | s') \nonumber \\
  &&=& \mathbb{I}(s'=s) \cdot d^{\pi_t}(s') \Big( \mathbb{I}(a'=a) - p^{\pi_\theta}(a | s') \Big) p^{\pi_\theta}(a' | s') \\
  \Rightarrow \quad & A_{(s, :), (s, :)} &=& d^{\pi_t}(s) \Big( \text{diag} (p^{\pi_\theta}(\cdot | s)) - p^{\pi_\theta}(\cdot | s) p^{\pi_\theta}(\cdot | s)^\top \Big),
\end{IEEEeqnarray}
where $p^{\pi_\theta}(\cdot | s) \in \mathbb{R}^{|\mathcal{A}|}$ is the vector defined as $[p^{\pi_\theta}(\cdot | s)]_a = p^{\pi_\theta}(a | s)$.

\section{PPO}
The PPO (Schulman et al., 2017) solves the following optimization problem at each iteration step:
\begin{equation}
  \max_\theta \; \underbrace{\sum_s d^{\pi_t}(s) \sum_a p^{\pi_t}(a | s) \cdot \min \left( \begin{matrix} \frac{p^{\pi_\theta}(a | s)}{p^{\pi_t}(a | s)} A^{\pi_t}(s, a), \\ \text{clip} \left[\frac{p^{\pi_\theta}(a | s)}{p^{\pi_t}(a | s)}, 1 - \epsilon, 1 + \epsilon \right] A^{\pi_t}(s, a) \end{matrix} \right)}_{=: \mathcal{J}}.
\end{equation}
The gradient of the objective $\mathcal{J}$ can be shown to be equivalent to
\begin{equation}
  \nabla \mathcal{J} = \sum_s d^{\pi_t}(s) \sum_a p^{\pi_t}(a | s) \cdot \mathbb{I} \Big( \text{cond}(s, a) \Big) \frac{\nabla p^{\pi_\theta}(a | s)}{p^{\pi_t}(a | s)} A^{\pi_t}(s, a),
\end{equation}
where 
\begin{equation}
  \text{cond}(s, a) = \left( A^{\pi_t}(s, a) > 0 \;\bigwedge\; \frac{p^{\pi_\theta}(a | s)}{p^{\pi_t}(a | s)} < 1 + \epsilon \right) \;\bigvee\; \left( A^{\pi_t}(s, a) < 0 \;\bigwedge\; \frac{p^{\pi_\theta}(a | s)}{p^{\pi_t}(a | s)} > 1 - \epsilon \right).
\end{equation}
Repeating our usual drill, we assume a softmax policy to obtain:
\begin{align}
  & \frac{\partial}{\partial \theta(s, a)} \mathcal{J} \nonumber \\
  &= \sum_x d^{\pi_t}(x) \sum_b \mathbb{I} \Big( \text{cond}(x, b) \Big) \frac{\partial p^{\pi_\theta}(b | x)}{\partial \theta(s, a)} A^{\pi_t}(x, b) \nonumber \\
  &= \sum_x d^{\pi_t}(x) \sum_b \mathbb{I} \Big( \text{cond}(x, b) \Big) \mathbb{I}(x = s) \Big( \mathbb{I}(b = a) - p^{\pi_\theta}(a | x) \Big) p^{\pi_\theta}(b | x) A^{\pi_t}(x, b) \nonumber \\
  &= d^{\pi_t}(s) \Bigg[ \sum_b \mathbb{I}(b = a) \mathbb{I} \Big( \text{cond}(s, b) \Big) p^{\pi_\theta}(b | s) A^{\pi_t}(s, b) - p^{\pi_\theta}(a | s) \sum_b \mathbb{I} \Big( \text{cond}(s, b) \Big) p^{\pi_\theta}(b | s) A^{\pi_t}(s, b) \Bigg] \nonumber \\
    &= d^{\pi_t}(s) p^{\pi_\theta}(a | s) \left[ \mathbb{I} \Big( \text{cond}(s, a) \Big) A^{\pi_t}(s, a) - p^{\pi_\theta}(a | s) \sum_b p^{\pi_\theta}(b | s) \mathbb{I} \Big( \text{cond}(s, b) \Big) A^{\pi_t}(s, b) \right]. \label{eq: ppo_gradient}
\end{align}
The PPO gradient (Eq. \ref{eq: ppo_gradient}) is exactly the same as the TRPO gradient (Eq. \ref{eq: trpo_gradient}) except for the additional condition on choosing only specific state-action pairs while calculating the difference between advantage under the current policy and the approximate change in advantage under the updated policy.

\section*{References}

\medskip
\small
\begin{list}{}{%
    \setlength{\topsep}{0pt}%
    \setlength{\leftmargin}{0.2in}%
    \setlength{\listparindent}{-0.2in}%
    \setlength{\itemindent}{-0.2in}%
    \setlength{\parsep}{\parskip}%
  }%

\item[] Beck, A., Teboulle, M. (2003). Mirror descent and nonlinear projected subgradient methods for convex optimization. \textit{Operations Research Letters, 31}(3), 167-175.

\item[] Nocedal, J., Wright, S. (2006). Numerical optimization. \textit{Springer Science \& Business Media.}

\item[] Schulman, J., Wolski, F., Dhariwal, P., Radford, A., Klimov, O. (2017). Proximal policy optimization algorithms. \textit{arXiv preprint arXiv:1707.06347.}

\item[] Schulman, J., Levine, S., Abbeel, P., Jordan, M., Moritz, P. (2015, June). Trust region policy optimization. In \textit{International conference on machine learning} (pp. 1889-1897). PMLR.
  
\item[] Vaswani, S., Bachem, O., Totaro, S., Mueller, R., Geist, M., Machado, M. C., Castro P. S., Roux, N. L. (2021). A functional mirror ascent view of policy gradient methods with function approximation. \textit{arXiv preprint arXiv:2108.05828.}
  
\end{list}

\end{document}
