%% if some package prevents compilation, remove it

\documentclass[a4paper, 11pt]{article}
\usepackage{fullpage}
\usepackage{titling}
\usepackage{graphicx}
\usepackage{xcolor}
\usepackage{amsmath}
\usepackage{amssymb}
\usepackage{amsthm}
\usepackage{hyperref}
\usepackage{IEEEtrantools} 
\usepackage{bbm}
\usepackage{lineno}
\usepackage{multirow}
\usepackage{longtable}
\usepackage{makecell}
\usepackage{lipsum}
\usepackage{etoolbox} %% <- for \pretocmd and \apptocmd
\usepackage{algorithm}
\usepackage[noend]{algpseudocode}

\setlength{\droptitle}{-6em}   % This is your set screw, for shifting up the title
\renewcommand{\baselinestretch}{1.1}
\setlength{\parskip}{0.5em}

\author{Shivam and Sharan
  \\ September 2021}
\date{}
\title{FMA-PG Notes}

\makeatletter %% <- make @ usable in macro names
\newcommand*\linenomathpatch{\@ifstar{\linenomathpatch@AMS}{\linenomathpatch@}}
\newcommand*\linenomathpatch@[1]{
  \expandafter\pretocmd\csname #1\endcsname {\linenomathWithnumbers}{}{}
  \expandafter\pretocmd\csname #1*\endcsname{\linenomathWithnumbers}{}{}
  \expandafter\apptocmd\csname end#1\endcsname {\endlinenomath}{}{}
  \expandafter\apptocmd\csname end#1*\endcsname{\endlinenomath}{}{}
}
\newcommand*\linenomathpatch@AMS[1]{
  \expandafter\pretocmd\csname #1\endcsname {\linenomathWithnumbersAMS}{}{}
  \expandafter\pretocmd\csname #1*\endcsname{\linenomathWithnumbersAMS}{}{}
  \expandafter\apptocmd\csname end#1\endcsname {\endlinenomath}{}{}
  \expandafter\apptocmd\csname end#1*\endcsname{\endlinenomath}{}{}
}
\let\linenomathWithnumbersAMS\linenomathWithnumbers
\patchcmd\linenomathWithnumbersAMS{\advance\postdisplaypenalty\linenopenalty}{}{}{}
\makeatother %% revert @
\linenomathpatch{IEEEeqnarray}
\linenomathpatch{equation}
\linenomathpatch*{gather}
\linenomathpatch*{multline}
\linenomathpatch*{align}
\linenomathpatch*{alignat}
\linenomathpatch*{flalign}

\begin{document}
\maketitle
\vspace{-2cm}
\linenumbers

\section{Softmax PPO Closed Form Update}
We will consider direct functional representation with tabular parameterization, i.e. $\pi \equiv p^\pi$ is essentially an $\mathrm{S} \times \mathrm{A}$ table satisfying the constraints
\begin{IEEEeqnarray*}{rCll}
  \sum_a p^\pi(a | s) &=& 1, & \qquad \forall s \in \mathcal{S} \\
  p^\pi(a | s) &\geq& 0, & \qquad \forall s \in \mathcal{S}, \; \forall a \in \mathcal{A}.
\end{IEEEeqnarray*}
Our goal is to find the closed form solution to the following optimization problem (from Eq. 6, Sharan et al., 2021):
\begin{equation}
  \pi_{t+1} = \arg\max_{\pi \in \Pi} \left[ \sum_s d^{\pi_t}(s) \sum_a p^{\pi_t}(a | s) \left(A^{\pi_t}(s, a) + \frac{1}{\eta} \right) \log \frac{p^\pi(s, a)}{p^{\pi_t}(s, a)} \right], \label{eq: orig_problem}
\end{equation}
subject to the above constraints on $p^\pi$. The above equation is obtained by setting the mirror map to the weighted exponential function.

We begin by formulating this problem using Lagrange multipliers $\lambda_s$, $\lambda_{s, a}$ for all states $s$ and actions $a$ (and with a slight abuse of the notation, i.e. we write $\lambda_s$ for all the $\lambda_s$s, etc.):
\begin{IEEEeqnarray}{rCl}
  \mathcal{L}(p^\pi, \lambda_s, \lambda_{s, a}) &=& \sum_s d^{\pi_t}(s) \sum_a p^{\pi_t}(a | s) \left(A^{\pi_t}(s, a) + \frac{1}{\eta} \right) \log \frac{p^\pi(a | s)}{p^{\pi_t}(a | s)} \nonumber \\
  && - \sum_{s, a} \lambda_{s, a} p^\pi(a | s) - \sum_s \lambda_{s} \bigg( \sum_a p^\pi(a | s) - 1 \bigg).
\end{IEEEeqnarray}
KKT conditions for this problem are:
\begin{align}
  \nabla_{p^\pi(x, b)} \mathcal{L}(p^\pi, \lambda_s, \lambda_{s, a}) = 0, \qquad \forall x \in \mathcal{S}, \; \forall b \in \mathcal{A} \tag{C1} \label{eq: KKT1} \\
  \sum_a p^\pi(a | s) = 1, \qquad \forall s \in \mathcal{S} \tag{C2} \label{eq: KKT2} \\
  p^\pi(a | s) \geq 0,  \qquad \forall s \in \mathcal{S}, \; \forall a \in \mathcal{A} \tag{C3} \label{eq: KKT3} \\
  \lambda_s \geq 0, \qquad \forall s \in \mathcal{S} \tag{C4} \label{eq: KKT4} \\
  \lambda_{s} \bigg( \sum_a p^\pi(a | s) - 1 \bigg) = 0 \qquad \forall s \in \mathcal{S} \tag{C5} \label{eq: KKT5} \\
  \lambda_{s, a} p^\pi(a | s) = 0, \qquad \forall s \in \mathcal{S}, \; \forall a \in \mathcal{A}. \tag{C6} \label{eq: KKT6}
\end{align}

Let us now try to solve this system. Solving the first equation for arbitrary state-action pair $(x, b)$, gives us:
\begin{IEEEeqnarray}{lrCl}
  & \nabla_{p^\pi(b | x)} \mathcal{L}(p^\pi, \lambda_s, \lambda_{s, a}) &=& d^{\pi_t}(x) p^{\pi_t}(b|x) \left( A^{\pi_t}(x, b) + \frac{1}{\eta} \right) \frac{1}{p^\pi(b|x)} - \lambda_{x, b} - \lambda_x = 0 \nonumber \\
  \Rightarrow & p^\pi(b | x) &=& \frac{d^{\pi_t}(x) p^{\pi_t}(b|x) (1 + \eta A^{\pi_t}(x, b))}{\eta (\lambda_x + \lambda_{x, b})}. \label{eq: lagrangian_derivative_sppo}
\end{IEEEeqnarray}
Let us set 
\begin{equation}
  \lambda_{s, a} = 0, \qquad \forall s \in \mathcal{S}, \; \forall a \in \mathcal{A}.
\end{equation}
Combining Eq. \ref{eq: lagrangian_derivative_sppo} with the second KKT condition gives us
\begin{equation}
  \lambda_s = \frac{1}{\eta} \sum_a d^{\pi_t}(s) p^{\pi_t}(a|s) (1 + \eta A^{\pi_t}(s, a)).
\end{equation}
Therefore, with the additional assumption $d^{\pi_t}(s) > 0$, $p^\pi(a | s)$ becomes
\begin{equation}
  p^\pi(a | s) = \frac{p^{\pi_t}(a|s) (1 + \eta A^{\pi_t}(s, a))}{\sum_b p^{\pi_t}(b|s) (1 + \eta A^{\pi_t}(s, b))}.
\end{equation}
Note that $d^{\pi_t}(s), p^{\pi_t}(a|s) \geq 0$ for any state-action pair, since they are proper measures. All that remains is to ensure that
\begin{equation*}
  1 + \eta A^{\pi_t}(s, a) \geq 0
\end{equation*}
to satisfy the third and fourth KKT conditions. But how to do that? One straightforward way is to define $p^\pi(a | s) = 0$ whenever $1 + \eta A^{\pi_t}(s, a) \leq 0$, and accordingly re-define $\lambda_s$. This gives us the final solution to our original optimization problem (Eq. \ref{eq: orig_problem}):
\begin{equation}
  \pi_{t+1} = p^\pi(s, a) = \frac{p^{\pi_t}(a|s) \max(1 + \eta A^{\pi_t}(s, a), 0)}{\sum_b p^{\pi_t}(b|s) \max(1 + \eta A^{\pi_t}(s, b), 0)}.
\end{equation}
\textbf{This leaves one last problem:} Is it always true that given any state $s$, there exists atleast one action $a$, such that $1 + \eta A^{\pi_t}(s, a) \geq 0$? Because otherwise, we would fail to satisfy the second KKT condition. We can possibly put another condition on $\eta$ in order to fulfill this condition.

\section{MDPO Closed Form Update}
The paper (Sharan et al., 2021) considers the direct representation along with tabular parameterization of the policy, albeit with a small change in notation as compared to the previous section: $\pi(a|s) \equiv p^\pi(a|s, \theta)$. However, since this notation is more cumbersome, we will stick to our old one: $\pi(a|s) \equiv p^\pi(a|s)$. The constraints on these parameters are the same as before: $\sum_a p^\pi(a | s) = 1, \; \forall s \in \mathcal{S}$; and $p^\pi(a | s) \geq 0, \; \forall s \in \mathcal{S}, \; \forall a \in \mathcal{A}$. Our goal, this time, is to solve the following optimization problem (see Eq. 9, Sharan et al., 2021)
\begin{equation}
  \pi_{t+1} = \arg\max_{\pi \in \Pi} \left[ \sum_s d^{\pi_t}(s) \sum_a p^{\pi_t}(a|s) \left( Q^{\pi_t}(s, a) \frac{p^\pi(a | s)}{p^{\pi_t}(a | s)} - \frac{1}{\eta} D_\phi (p^\pi(\cdot | s), p^{\pi_t}(\cdot | s)) \right) \right],
\end{equation}
with the mirror map as the negative entropy (Eq. 5.27, Beck and Teboulle, 2002). This particular choice results in
\begin{equation}
  D_\phi (p^\pi(\cdot | s), p^{\pi_t}(\cdot | s)) = \text{KL}(p^\pi(\cdot | s) \| p^{\pi_t}(\cdot | s)) := \sum_a p^\pi(a | s) \log \frac{p^\pi(a | s)}{p^{\pi_t}(a | s)}.
\end{equation}
The optimization problem then simplifies to
\begin{equation}
  \pi_{t+1} = \arg\max_{\pi \in \Pi} \left[ \sum_s d^{\pi_t}(s) \sum_a p^{\pi_t}(a|s) \left( Q^{\pi_t}(s, a) \frac{p^\pi(a | s)}{p^{\pi_t}(a | s)} - \frac{1}{\eta} \sum_{a'} p^\pi(a' | s) \log \frac{p^\pi(a' | s)}{p^{\pi_t}(a' | s)} \right) \right].
\end{equation}

Proceeding analogously to the previous section, we use Lagrange multipliers $\lambda_s$, $\lambda_{s, a}$ for all states $s$ and actions $a$ to obtain the function
\begin{IEEEeqnarray}{rCl}
  \mathcal{L}(p^\pi, \lambda_s, \lambda_{s, a}) &=& \sum_s d^{\pi_t}(s) \sum_a p^{\pi_t}(a|s) Q^{\pi_t}(s, a) \frac{p^\pi(a | s)}{p^{\pi_t}(a | s)} - \frac{1}{\eta} \sum_s d^{\pi_t}(s) \sum_{a'} p^\pi(a' | s) \log \frac{p^\pi(a' | s)}{p^{\pi_t}(a' | s)} \nonumber \\
  && - \sum_{s, a} \lambda_{s, a} p^\pi(a | s) - \sum_s \lambda_{s} \bigg( \sum_a p^\pi(a | s) - 1 \bigg).
\end{IEEEeqnarray}
The KKT conditions are exactly the same as before (Eq. \ref{eq: KKT1} to Eq. \ref{eq: KKT6}).

Again, we begin by solving the first KKT condition:
\begin{IEEEeqnarray}{lrCl}
  & \nabla_{p^\pi(b | x)} \mathcal{L}(p^\pi, \lambda_s, \lambda_{s, a}) &=& d^{\pi_t}(x) p^{\pi_t}(b|x) \frac{Q^{\pi_t}(x, b)}{p^{\pi_t}(b | x)} - \frac{d^{\pi_t}(x)}{\eta} \left[ \log \frac{p^\pi(b | x)}{p^{\pi_t}(b | x)} + 1 \right] - \lambda_{x, b} - \lambda_x \nonumber \\
  &&=& \frac{d^{\pi_t}(x)}{\eta} \left[ \eta Q^{\pi_t}(x, b) - \log \frac{p^\pi(b | x)}{p^{\pi_t}(b | x)} - 1 - \frac{\eta (\lambda_{x, b} + \lambda_x)}{d^{\pi_t}(x)} \right] \nonumber \\
  &&=& 0 \nonumber \\
  \Rightarrow & \log \frac{p^\pi(b | x)}{p^{\pi_t}(b | x)} &=& \eta Q^{\pi_t}(x, b) - \frac{\eta (\lambda_{x, b} + \lambda_x)}{d^{\pi_t}(x)} - 1 \nonumber \\
  \Rightarrow & p^\pi(b | x) &=& p^{\pi_t}(b | x) \cdot \exp ( \eta Q^{\pi_t}(x, b) ) \cdot \exp \left(- \frac{\eta (\lambda_{x, b} + \lambda_x)}{d^{\pi_t}(x)} - 1 \right), \label{eq: lagrangian_derivative_mdpo}
\end{IEEEeqnarray}
where in the fourth line, we made the assumption that $d^{\pi_t}(x) > 0$ for all states $x$. We again set
\begin{equation}
  \lambda_{s, a} = 0, \qquad \forall s \in \mathcal{S}, \; \forall a \in \mathcal{A}.
\end{equation}
Yet again, we put Eq. \ref{eq: lagrangian_derivative_mdpo} in the second KKT condition to get
\begin{equation}
  \exp \left(- \frac{\eta \lambda_x}{d^{\pi_t}(x)} - 1 \right) = \left( \sum_b p^{\pi_t}(b | x) \cdot \exp ( \eta Q^{\pi_t}(x, b) ) \right)^{-1}.
\end{equation}
Therefore, we obtain
\begin{equation}
  p^\pi(a | s) = \frac{p^{\pi_t}(a | s) \cdot \exp ( \eta Q^{\pi_t}(s, a) )}{\sum_b p^{\pi_t}(b | s) \cdot \exp ( \eta Q^{\pi_t}(s, b) )}.
\end{equation}

\textbf{Again, this leaves one last problem:} Can we ensure that $\lambda_s \geq 0$ for all states $s$? If not, then the fourth KKT condition cannot be satisfied. Maybe, we can set the stepsize $\eta$ in such a way, such that this constraint is always fulfilled.


\section*{References}

\medskip
\small
\begin{list}{}{%
    \setlength{\topsep}{0pt}%
    \setlength{\leftmargin}{0.2in}%
    \setlength{\listparindent}{-0.2in}%
    \setlength{\itemindent}{-0.2in}%
    \setlength{\parsep}{\parskip}%
  }%

\item[] Beck, A., Teboulle, M. (2003). Mirror descent and nonlinear projected subgradient methods for convex optimization. \textit{Operations Research Letters, 31}(3), 167-175.
  
\item[] Vaswani, S., Bachem, O., Totaro, S., Mueller, R., Geist, M., Machado, M. C., Castro P. S., Roux, N. L. (2021). A functional mirror ascent view of policy gradient methods with function approximation. \textit{arXiv preprint arXiv:2108.05828.}
  
\end{list}

\end{document}
